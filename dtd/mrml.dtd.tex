
%TESTBEGIN
\begin{verbatim}<!--\end{verbatim}
  
       \paragraph{Basic structure:}
       {\em
       Messages are sent as MRML texts.
       In  order to make it easy for the server to know who connects,
       each message is assigned the id of its session as an attribute.
       }

     \begin{center}
      Author of this file: 
      Wolfgang Mueller with lots of suggestions and \\
                                   corrections from \\
      David Squire, Arjen P. de Vries and Christoph Giess \\
     \end{center}
       
 \begin{verbatim}-->\end{verbatim}



\begin{verbatim}
<!ELEMENT mrml (begin-transaction?,(get-configuration|configuration-description|get-sessions|session-list|open-session|rename-session|close-session|delete-session|get-collections|collection-list|get-algorithms|algorithm-list|get-property-sheet|property-sheet|configure-session|query-step|query-result|user-data|error)?,end-transaction?) 
>\end{verbatim}

\begin{verbatim}
<!ATTLIST mrml 
                 session-id ID #IMPLIED
                 transaction-id ID #IMPLIED
>\end{verbatim}

\begin{verbatim}<!--\end{verbatim}
  



\requesttitle{get-configuration}

This is the message an MRML client sends to the server on connection.
The message \mrmltag{get-configuration} gives information about the basic server configuration.

       
 \begin{verbatim}-->\end{verbatim}



\begin{verbatim}
<!ELEMENT get-configuration EMPTY 
>\end{verbatim}

\begin{verbatim}<!--\end{verbatim}
  


\responsetitle{configuration-description}

The \mrmltag{get-configuration} message is anwered by  a  message which
is supposed to hold a description about anything which is nonstandard \MRML.


       
 \begin{verbatim}-->\end{verbatim}



\begin{verbatim}
<!ELEMENT configuration-description (error?) 
>\end{verbatim}

\begin{verbatim}<!--\end{verbatim}
   

     \requesttitle{get-sessions}

     The \mrmltag{get-sessions} message permits the user to request
     exsisting sessions for a given user. It is sent by the client
     directly after after the \mrmltag{configuration-description} has
     been delivered, and prior to any other activity. 

     Authentification happens before any other activity to give the 
     server the possibility to customise any other information sent
     to the user. For example, it might be sensible to send different
     property sheets to different classes of users, or to render some
     parts of the database only visible to the own work group.
     
       
 \begin{verbatim}-->\end{verbatim}



\begin{verbatim}
<!ELEMENT get-sessions EMPTY 
>\end{verbatim}

\begin{verbatim}
<!ATTLIST get-sessions 
                         user-name CDATA #REQUIRED
                         password CDATA 'guest'
>\end{verbatim}

\begin{verbatim}<!--\end{verbatim}
  

     \responsetitle{session-list}
     
     A session is described by its \mrmltag{session-id}. 
     We also send over a more human-readable name


       
 \begin{verbatim}-->\end{verbatim}



\begin{verbatim}
<!ELEMENT session-list (session+|error) 
>\end{verbatim}

\begin{verbatim}
<!ELEMENT session (error?) 
>\end{verbatim}

\begin{verbatim}
<!ATTLIST session 
                    session-id CDATA #REQUIRED
                    session-name CDATA 'Default session'
>\end{verbatim}

\begin{verbatim}<!--\end{verbatim}
   

     \requesttitle{get-collections}

     gets a list of collections on the server.

        
 \begin{verbatim}-->\end{verbatim}



\begin{verbatim}
<!ELEMENT get-collections EMPTY 
>\end{verbatim}

\begin{verbatim}<!--\end{verbatim}
    
  
     \requesttitle{collection-list}

     a list of collections on the server.

     a collection is described by a list of the
     of the query paradigms which can be used
     for querying it, as well as an ID and its
     human-readable name.

     This means, you do not have to index all collections using all
     the methods you want to propose to the server.

        
 \begin{verbatim}-->\end{verbatim}



\begin{verbatim}
<!ELEMENT collection-list (collection*|error) 
>\end{verbatim}

\begin{verbatim}
<!ELEMENT collection (query-paradigm-list?|error) 
>\end{verbatim}

\begin{verbatim}
<!ATTLIST collection 
                       collection-id CDATA #REQUIRED
                       collection-name CDATA #REQUIRED
>\end{verbatim}

\begin{verbatim}<!--\end{verbatim}
  

     \tagtitle{query-paradigm}
     arises both in algorithms and collections:
     this describes the kind of query which can
     be performed with this algorithm/collection 

       
 \begin{verbatim}-->\end{verbatim}



\begin{verbatim}
<!ELEMENT query-paradigm-list (query-paradigm*|error) 
>\end{verbatim}

\begin{verbatim}
<!ELEMENT query-paradigm (error?) 
>\end{verbatim}

\begin{verbatim}
<!ATTLIST query-paradigm 
                           query-paradigm-id CDATA #REQUIRED
>\end{verbatim}

\begin{verbatim}<!--\end{verbatim}
  

     \requesttitle{get-algorithms}

     gets a list of algorithms usable for one collection  

       
 \begin{verbatim}-->\end{verbatim}



\begin{verbatim}
<!ELEMENT get-algorithms EMPTY 
>\end{verbatim}

\begin{verbatim}
<!ATTLIST get-algorithms 
                           collection-id CDATA #IMPLIED
                           query-paradigm-id CDATA #IMPLIED
>\end{verbatim}

\begin{verbatim}<!--\end{verbatim}
   

     \responsetitle{algorithm-list}

     returns a list of algorithms for a given collection on the server 

       
 \begin{verbatim}-->\end{verbatim}



\begin{verbatim}
<!ELEMENT algorithm-list (algorithm*|error) 
>\end{verbatim}

\begin{verbatim}<!--\end{verbatim}
   

     \tagtitle{algorithm}
     An algorithm can contain 
     other algorithms,
     optionally a property sheet,
     optionally a query paradigm list
     optionally an "allows-children" specification

       
 \begin{verbatim}-->\end{verbatim}



\begin{verbatim}
<!ELEMENT algorithm ((algorithm*,property-sheet?,query-paradigm-list?,allows-children?)|error) 
>\end{verbatim}

\begin{verbatim}
<!ATTLIST algorithm 
                      algorithm-id CDATA #REQUIRED
                      collectiom-id CDATA #REQUIRED
                      algorithm-name CDATA #REQUIRED
                      algorithm-id ID #REQUIRED
                      algorithm-type CDATA 'adefault'
                      collection-id CDATA 'cdefault'
>\end{verbatim}

\begin{verbatim}<!--\end{verbatim}
   

     \tagtitle{allows-children}
     This tag specifies for an algorithm what kind of algorithms
     can be children of this algorithm. no specification $\Rightarrow$ allows no
     children.

       
 \begin{verbatim}-->\end{verbatim}



\begin{verbatim}
<!ELEMENT allows-children ((query-paradigm-list?)|error) 
>\end{verbatim}

\begin{verbatim}<!--\end{verbatim}
   

     \requesttitle{get-property-sheet}
     get the property sheet for an algorithm

       
 \begin{verbatim}-->\end{verbatim}



\begin{verbatim}
<!ELEMENT get-property-sheet EMPTY 
>\end{verbatim}

\begin{verbatim}
<!ATTLIST get-property-sheet 
                               algorithm-id ID #REQUIRED
>\end{verbatim}

\begin{verbatim}<!--\end{verbatim}
   

     \requesttitle{begin-transaction}
     begins a transaction 

       
 \begin{verbatim}-->\end{verbatim}



\begin{verbatim}
<!ATTLIST begin-transaction 
                              transaction-id ID #REQUIRED
>\end{verbatim}

\begin{verbatim}<!--\end{verbatim}
   

     \requesttitle{end-transaction}
     ends a transaction 

       
 \begin{verbatim}-->\end{verbatim}



\begin{verbatim}
<!ATTLIST end-transaction 
                            transaction-id ID #REQUIRED
>\end{verbatim}

\begin{verbatim}<!--\end{verbatim}
  

     \requesttitle{configure-session}
     configures the session by giving an algorithm     

      
 \begin{verbatim}-->\end{verbatim}



\begin{verbatim}
<!ELEMENT configure-session (algorithm) 
>\end{verbatim}

\begin{verbatim}<!--\end{verbatim}
  

       \tagtitle{property-sheet}

       \textbf{Basic idea}: \parbox[t]{5cm}{send a property sheet together with a specification
                                            what should be the XML output coming back.
                                            Useful for configuring your database.}
       
 \begin{verbatim}-->\end{verbatim}



\begin{verbatim}
<!ELEMENT property-sheet ((property-sheet)*|error) 
>\end{verbatim}

\begin{verbatim}
<!ATTLIST property-sheet 
                           property-sheet-id ID #REQUIRED
                           property-sheet-type (multi-set|subset|set-element|boolean|numeric|textual|panel|clone|reference) #REQUIRED
                           caption CDATA #IMPLIED
                           visibility (popup|visible|invisible) 'visible'
                           send-type (element|attribute|attribute-name|attribute-value|children|none) #REQUIRED
                           send-name CDATA #IMPLIED
                           send-value CDATA #IMPLIED
                           min-subset-size CDATA #IMPLIED
                           max-subset-size CDATA #IMPLIED
                           from CDATA #IMPLIED
                           step CDATA #IMPLIED
                           to CDATA #IMPLIED
                           auto-id (yes|no) #IMPLIED
                           auto-id-name CDATA 'id'
                           defaultstate CDATA #IMPLIED
>\end{verbatim}

\begin{verbatim}<!--\end{verbatim}
  

     \tagtitle{algorithm}
     An algorithm will be either an empty element with 
     attributes (add some at your will, it will talk with your 
     server anyway, and this is the server which sent the property 
     sheet), or a tree of algorithms.

     What is the use of this?
     Think of configuring meta queries. Together with properties
     you get a powerful tool.

       
 \begin{verbatim}-->\end{verbatim}



\begin{verbatim}
<!ELEMENT algorithm ((algorithm*)|error) 
>\end{verbatim}

\begin{verbatim}<!--\end{verbatim}
  

     \textbf{Beginning and renaming sessions}

     We want to give the user the possibility to save the current
     state into "sessions". This might be useful in the case that 
     a user has several classes of goals which s/he knows in advance.

     The user can request a new session.
     S/he can also request a name change for a session.
     
     Ending sessions is implicit:
     we cannot afford being dependent on the user ending
     his/her session in a "nice" way, so we do not 
     tempt programmers to do so 

       
 \begin{verbatim}-->\end{verbatim}



\begin{verbatim}<!--\end{verbatim}
  Interface side  
 \begin{verbatim}-->\end{verbatim}



\begin{verbatim}<!--\end{verbatim}
  send the desired feedback method together with 
       a name for the session  
 \begin{verbatim}-->\end{verbatim}



\begin{verbatim}
<!ELEMENT open-session EMPTY 
>\end{verbatim}

\begin{verbatim}
<!ATTLIST open-session 
                         user-name CDATA #REQUIRED
                         password CDATA #IMPLIED
                         session-id CDATA #IMPLIED
                         session-name CDATA #IMPLIED
>\end{verbatim}

\begin{verbatim}
<!ELEMENT rename-session EMPTY 
>\end{verbatim}

\begin{verbatim}
<!ATTLIST rename-session 
                           session-id CDATA #IMPLIED
                           session-name CDATA #IMPLIED
>\end{verbatim}

\begin{verbatim}
<!ELEMENT delete-session EMPTY 
>\end{verbatim}

\begin{verbatim}
<!ATTLIST delete-session 
                           session-id CDATA #REQUIRED
>\end{verbatim}

\begin{verbatim}
<!ELEMENT close-session EMPTY 
>\end{verbatim}

\begin{verbatim}
<!ATTLIST close-session 
                          session-id CDATA #REQUIRED
>\end{verbatim}

\begin{verbatim}
<!ELEMENT acknowledge-session-op EMPTY 
>\end{verbatim}

\begin{verbatim}
<!ATTLIST acknowledge-session-op 
                                   session-id CDATA #REQUIRED
>\end{verbatim}

\begin{verbatim}<!--\end{verbatim}
  

     \textbf{Simple user commands (for logging purposes)}

     (like e.g. back or forward 
         in the command history)
     (at present the only commands)
      
 \begin{verbatim}-->\end{verbatim}



\begin{verbatim}
<!ELEMENT user-data EMPTY 
>\end{verbatim}

\begin{verbatim}
<!ATTLIST user-data 
                      command (history-backward|history-forward) 'backward'
                      steps CDATA #IMPLIED
>\end{verbatim}

\begin{verbatim}<!--\end{verbatim}
  

     \requesttitle{query-step}

     At present we provide only query by example, and search for random
     images (done, if one sends an empty \mrmltag{query-step} tag)


       
 \begin{verbatim}-->\end{verbatim}



\begin{verbatim}
<!ELEMENT query-step ((user-relevance-element-list|)*) 
>\end{verbatim}

\begin{verbatim}
<!ATTLIST query-step 
                       query-step-id CDATA #IMPLIED
                       result-size CDATA #IMPLIED
                       result-cutoff CDATA #IMPLIED
                       query-type (query|at-random) 'query'
                       algorithm-id CDATA #IMPLIED
>\end{verbatim}

\begin{verbatim}<!--\end{verbatim}
  \tagtitle{user-relevance-element-list}

     List of URLs with user given relevances 
     Our way of specifying a QBE for images.

     relevances vary from 0 to 1
      
 \begin{verbatim}-->\end{verbatim}



\begin{verbatim}
<!ELEMENT user-relevance-element-list (user-relevance-element+) 
>\end{verbatim}

\begin{verbatim}
<!ELEMENT user-relevance-element EMPTY 
>\end{verbatim}

\begin{verbatim}
<!ATTLIST user-relevance-element 
                                   user-relevance CDATA #REQUIRED
                                   image-location CDATA #REQUIRED
>\end{verbatim}

\begin{verbatim}<!--\end{verbatim}
  

     \responsetitle{query-result}

     each result image can be accompanied by the user given relevance,
     as well as the similarity calculated by the program, based on the
     feature space.

     calculated similarities vary from 0 to 1

       
 \begin{verbatim}-->\end{verbatim}



\begin{verbatim}
<!ELEMENT query-result ((query-resultelement-list*,query-result*)|error) 
>\end{verbatim}

\begin{verbatim}
<!ELEMENT query-result-element-list ((query-result-element|error)+) 
>\end{verbatim}

\begin{verbatim}
<!ELEMENT query-result-element (error?) 
>\end{verbatim}

\begin{verbatim}
<!ATTLIST query-result-element 
                                 calculated-similarity CDATA #REQUIRED
                                 thumbnail-location CDATA #REQUIRED
                                 image-location CDATA #REQUIRED
>\end{verbatim}

\begin{verbatim}<!--\end{verbatim}
  

     Error messages.

       
 \begin{verbatim}-->\end{verbatim}



\begin{verbatim}
<!ELEMENT error EMPTY 
>\end{verbatim}

\begin{verbatim}
<!ATTLIST error 
                  message CDATA #REQUIRED
>\end{verbatim}

\begin{verbatim}
<!ATTLIST cui-time-stamp 
                           calendar-time CDATA #REQUIRED
>\end{verbatim}

\begin{verbatim}<!--\end{verbatim}
  \tagtitle{cui-text-query}
        query using a number of keywords in a string
      
 \begin{verbatim}-->\end{verbatim}



\begin{verbatim}
<!ATTLIST cui-text-query 
                           query-string CDATA #IMPLIED
>\end{verbatim}


%TESTEND
