\documentclass[10pt]{article}
\usepackage{xspace} 
\usepackage{amstex}
\usepackage{a4}
\usepackage{epsfig}
\usepackage{cite}
% Im Moment leer

%Marking a text passage as ToDo
\newcommand{\ToDo}[1]{\marginpar{{\tt ToDo}}\underline{#1}}

\newcommand{\interval}[1]{{\iota_{#1}}}
\newcommand{\discrete}{\delta}
\newcommand{\continuous}{\varphi}
\newcommand{\continuousVec}{{\vec\varphi}}
\newcommand{\sdev}{\sigma}
\newcommand{\GaussF}[1]{G_{#1}}
\newcommand{\Gauss}[2]{\GaussF{#1}\left({#2}\right)}
\newcommand{\GaussFSdev}{\GaussF{\sdev}}
\newcommand{\etal}{{\em et al.}}
\newcommand{\mybox}[2]{
\begin{figure}
\epsfig{file=#1.eps,width=12cm}
\caption{\label{#1}{\em #2}}
\end{figure}

}
\newcommand{\strich}{^\prime}

\newcommand{\ImageToDiscreteFeatures}{\Lambda}
\newcommand{\CumulativeDistributionFunction}{\Phi}
\newcommand{\eg}{\emph{e.g.}\@\xspace} 
\newcommand{\ie}{\emph{i.e.}\@\xspace} 
\newcommand{\etc}{\emph{etc.}\@\xspace} 
\newcommand{\ldq}{\lq\lq}
\newcommand{\rdq}{\rq\rq\xspace}


\newcommand{\df}{\mathrm{df}}
\newcommand{\cf}{\mathrm{cf}}
\newcommand{\wf}{\mathrm{wf}}
\newcommand{\nf}{\mathrm{nf}}
\newcommand{\tf}{\mathrm{tf}}
\newcommand{\lf}{\mathrm{lcf}}
\newcommand{\abs}{\mathrm{abs}}


\begin{document}

\title{GIFT's current weighting schemes}
\author{Wolfgang M\"uller}

\maketitle

For the combination of query image weights GIFT uses the formula:
\[
	\df_{qj}=\sum_{i=1}^N \df_{ij}\cdot R_i
\]

I define the normalization factor:
\[
	\nf:=\sum_i \abs(R_i);
\]

For scoring GIFT uses:
\[
	s_{k_{new}}=s_{k_{old}} + \wf(q,k,j)
\]

$\wf$ is here the weighting function. Before giving a table of
weighting functions I define some helper functions which depend on the
collection frequency of an image and  the kind of feature (block or
histogram) as well as on the actual query and the actual document.

The standard factor for a document $k$, 
query $q$ and feature $j$:
\[
	\wf^{0}_{kqj}:=
	\begin{cases}	
		\frac{sgn(\tf_{qj})}{\nf}&\text{for block features}\\

		\frac 1{\nf}\cdot sgn(\tf_{qj})\cdot \min
		\left\{\abs\left(\frac{\tf_{qj}}{\nf}\right),
		\tf_{kj}\right\}& \text{for histogram features}\\

	\end{cases}
\]

There arise two different logarithmic factors of the collection frequency:
\[
	\lf_{1j}=\begin{cases}
		\log(\frac 1{\cf_j})& \text{for block features}\\
		1 & \text{for histogram features}\\
		\end{cases}
\]
and
\[
	\lf_{2j}=\begin{cases}
		\log(\frac 1{\cf_j}-1+\epsilon)& \text{for block features}\\
		1 & \text{for histogram features}\\
		\end{cases}
\]
where the $\epsilon$ is added to avoid overflows.

The weighting schemes presently coded are:

\begin{tabular}{|c|p{3cm}|l|}
\hline
2&Best weighted probabilistic weight & 
	$\left(0.5+\frac{0.5 \tf_{kj}}{\max_j \tf_{kj}}\right)\cdot\lf_{2j}\cdot 
	\wf^0_{qkj}$\\\hline
3&Classical idf weight & 
	$\left(\lf_{1j}\right)^2\cdot \wf^0_{qkj}$\\\hline
4&Binary term independence & $\lf_{2j}\cdot \wf^0_{qkj}$\\\hline
5&Standard tf weight& $\frac 1{\sqrt{\sum_i\tf_{qi}^2}}\cdot\wf^0_{qkj}\cdot
	\begin{cases}
		{\tf_{kj}\cdot\tf_{qj}}&\text{for block features}\\
		1 &\text{for historgram features}\\
	\end{cases}$\\\hline
6&Coordination level&$\wf^0_{qkj}$\\\hline
\end{tabular}


\end{document}
















